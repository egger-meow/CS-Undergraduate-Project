\documentclass[12pt,a4paper]{article}

% Chinese support
\usepackage[UTF8]{ctex}
\usepackage{xeCJK}

% Professional report packages
\usepackage[margin=2.5cm]{geometry}
\usepackage{amsmath,amsfonts,amssymb}
\usepackage{graphicx}
\usepackage{float}
\usepackage{booktabs}
\usepackage{array}
\usepackage{multirow}
\usepackage{longtable}
\usepackage{enumitem}
\usepackage{url}
\usepackage{hyperref}
\usepackage{fancyhdr}
\usepackage{titlesec}
\usepackage{color}
\usepackage{listings}
\usepackage{caption}
\usepackage{subcaption}

% Page style
\pagestyle{fancy}
\fancyhf{}
\rhead{\thepage}
\lhead{電梯異常檢測(再平層)之研究}
\renewcommand{\headrulewidth}{0.4pt}

% Title formatting
\titleformat{\section}{\Large\bfseries}{\thesection}{1em}{}
\titleformat{\subsection}{\large\bfseries}{\thesubsection}{1em}{}
\titleformat{\subsubsection}{\normalsize\bfseries}{\thesubsubsection}{1em}{}

% Hyperlink setup
\hypersetup{
    colorlinks=true,
    linkcolor=black,
    filecolor=magenta,      
    urlcolor=blue,
    citecolor=black
}

% Line spacing
\usepackage{setspace}
\onehalfspacing

\begin{document}

% Title page
\begin{titlepage}
    \centering
    \vspace*{2cm}
    
    {\Huge\bfseries 專題報告}\\[0.5cm]
    {\huge\bfseries 電梯異常檢測(再平層)之研究}\\[2cm]
    
    {\Large 陽明交通大學 管科系/資工系}\\[0.5cm]
    {\Large 侯均頲}\\[3cm]
    
    \vfill
    
    {\large \today}
\end{titlepage}

% Table of contents
\tableofcontents
\newpage

% Abstract/Overview
\section{專案概述}

\textbf{本研究基於參與電梯異常檢測專案的經驗,取用專案中學長姐蒐集的電梯實驗資料進行實驗與方法研究。}

\subsection{研究目標}
偵測「再平層」事件(電梯到站第一次對齊後,因誤差過大而進行第二次對齊),以降低人工巡檢負擔。

\subsection{資料來源}
使用專案中蒐集的馬達電流(channel 0)+ 六軸振動(門上 XYZ:channels 1–3;車廂/馬達背面 XYZ:channels 4–6)資料,原始取樣率 8192 Hz,訓練前下採樣至約 256 Hz;實驗使用 0604–0605 的正常資料與 0607 的異常資料;切片聚焦 \textbf{Stage 3–5}。

\subsection{研究方法}
各自以正常資料、異常資料訓練兩個 \textbf{1D-CNN Autoencoder};每筆樣本輸入後得到兩個 reconstruction loss(視作 2D 特徵),再用 SVM / Logistic Regression / kNN 做最後分類。

\subsection{通道發現}
門上 XYZ(1–3)較有訊號意義;車廂/馬達背面 XYZ(4–6)資訊性弱。電流(0)+ 門上振動(1,2,3)是最佳組合;包含 4–6 的組合準確率常掉到 \textasciitilde 0.7 或以下。

\subsection{融合策略}
電流模型與振動模型的輸出做 \textbf{bitwise OR},以小幅犧牲 precision 換取更高 recall(務實符合維運場景)。

\subsection{評估方法}
採 \textbf{10 次完整管線重複實驗}(每次重新隨機劃分 train/test),彙整平均表現。

\section{問題定義與資料描述}

\subsection{業務問題(再平層)}

\subsubsection{再平層定義}
電梯到達某一樓層後,控制系統會進行一次「平層」動作,將車廂位置與樓層對齊。但若\textbf{第一次平層誤差過大},電梯會再進行一次\textbf{第二次平層}(即「再平層」)。

\subsubsection{異常意義}
\begin{itemize}
    \item 再平層雖然不是立即性的安全事故,但它往往是\textbf{潛在問題的前兆}。
    \item 根據公司內部的實務經驗,\textbf{再平層的出現與後續的多種異常狀況(例如馬達老化、煞車失效、控制系統校正異常、門機故障)呈高度正相關}。
    \item 因此,\textbf{偵測再平層 ≈ 預警更多嚴重異常}。
\end{itemize}

\subsubsection{目前困境}
\begin{itemize}
    \item 公司原本依賴\textbf{人工檢查}(維修人員聽聲音或查看記錄)來判斷是否出現再平層。
    \item 人工檢查效率低、成本高,且受人員經驗影響。
\end{itemize}

\subsubsection{專題目標}
開發一套\textbf{基於感測訊號的自動化異常檢測方法},讓系統能在早期就發現再平層,進而預防或降低後續異常風險,並大幅減少人工巡檢成本。

\subsection{感測配置與通道}

\begin{itemize}
    \item \textbf{Channel 0}:馬達電流(amp)。
    \item \textbf{Channels 1–3}:門上振動 XYZ
    \item \textbf{Channels 4–6}:車廂/馬達背面振動 XYZ
\end{itemize}

基於實驗資料分析發現:\textbf{門上 XYZ 資訊性高,車廂/馬達背面 XYZ 資訊性低}。

x, y軸代表正常autoencoder與異常autoencoder的reconstruction loss (後續會提及),一個點代表一個test樣本 (一個樣本是一個時間序列)

\subsection{數據期間與採樣}

學長姐蒐集的資料期間與採樣設計:

\begin{itemize}
    \item 正常資料:\textbf{0604–0605};異常資料:\textbf{0607}。原始 \textbf{8192 Hz},訓練前下採樣至 \textbf{\textasciitilde 256 Hz}。
    \item 關注片段:\textbf{Stage 3–5}(對應行車階段中最能表徵再平層的訊段)。
\end{itemize}

\subsection{資料切分與數量}

基於學長姐蒐集的資料進行劃分,設定為 \textbf{Train/Validate = 85\%/15\%},並列出數量:

\begin{itemize}
    \item 正常:train \textbf{1240}、valid \textbf{50}、test \textbf{50}
    \item 異常:train \textbf{62}、valid \textbf{50}、test \textbf{50}
\end{itemize}

(註:異常樣本稀少,故另留固定數量的驗證與測試,以保持評估公平。)

\section{方法設計(Architecture \& Pipeline)}

\subsection{雙自編碼器(Dual-AE)設計}

\subsubsection{各自訓練兩個 1D-CNN Autoencoder:}

\begin{enumerate}
    \item \textbf{Normal-AE} (正常 autoencoder):只用正常資料訓練
    \item \textbf{Abnormal-AE} (異常 autoencoder):只用異常資料訓練
\end{enumerate}

推論時對每筆樣本分別計算 Normal-AE 與 Abnormal-AE 的 \textbf{reconstruction loss};把兩個 loss 視作 \textbf{2D 特徵向量}(x=normal-loss, y=abnormal-loss),再餵入分類器(\textbf{SVM / Logistic Regression / kNN})。

\section{通道與模型訓練策略}

\subsection{單通道建模}

\subsubsection{電流(Channel 0)}

使用馬達電流單通道資料,訓練一對 Autoencoder。
\textbf{實驗結果}:

\begin{itemize}
    \item \textbf{Precision 較高(0.94)},代表「判斷為異常時幾乎不會誤報」。
    \item 但 Recall 偏低(0.62),異常檢出率不足。
    \item 適合作為\textbf{高精確度但較保守}的檢測器。
\end{itemize}

\subsubsection{門上振動(Channels 1,2,3)}

使用門上振動三軸資料,訓練另一對 Autoencoder。
\textbf{實驗結果}:

\begin{itemize}
    \item \textbf{Recall 較高(0.84)},代表「異常檢出率較佳」。
    \item Precision 相對較低(0.82),誤報機率增加。
    \item 適合作為\textbf{高靈敏度但較寬鬆}的檢測器。
\end{itemize}

\subsubsection{電流 + 門上振動(0,1,2,3)}

\begin{itemize}
    \item 嘗試將電流與振動直接合併輸入,但結果並沒有比單獨訓練更好。
    \item 原因推測為通道間訊號特性差異過大,導致 AE 難以同時兼顧。
\end{itemize}

\subsection{後融合策略(OR)}

\subsubsection{由於單一模型各有優劣,採取 OR 融合:}

\begin{itemize}
    \item \textbf{OR 規則}:\texttt{pred\_final = pred\_amp OR pred\_vib}
    \item 即若任一模型判斷為異常,則最終輸出異常。
\end{itemize}

\subsubsection{優勢:}

\begin{itemize}
    \item Precision 稍微下降(0.82),但 Recall 大幅提升(0.92)。
    \item F1-Score 提升至 0.87,是三者最佳。
    \item 符合實務需求(寧可多報,不能漏報)。
\end{itemize}

\section{實驗與結果}

\subsection{結果數據表}

\begin{table}[H]
\centering
\caption{各模型性能比較}
\begin{tabular}{@{}lcccc@{}}
\toprule
模型 & Accuracy & Precision & Recall & F1-Score \\
\midrule
Amp (電流0) & 0.79 & \textbf{0.94} & 0.62 & 0.75 \\
Vib (1,2,3) & 0.83 & 0.82 & \textbf{0.84} & 0.83 \\
Mix (OR) & \textbf{0.86} & 0.82 & \textbf{0.92} & \textbf{0.87} \\
\bottomrule
\end{tabular}
\end{table}

\textbf{結論}:電流(0)側重 Precision,門上振動(1,2,3)側重 Recall,兩者互補。透過 OR 融合,整體 F1 最佳,Recall 亦達到實務所需的水準。

\subsection{單路結果(每路均為 Dual-AE → 2D loss → 分類器)}

\begin{itemize}
    \item \textbf{電流(0)}:完成 train/test 評估(投影片給出混淆矩陣式標示 0/1)。
    \item \textbf{門上振動(1,2,3)}:同上,完成 train/test 評估。
\end{itemize}

\subsection{通道組合與表現}

\begin{itemize}
    \item \textbf{最佳}:\texttt{[vibration 1,2,3] + [amp 0]}
    \item \textbf{次佳或不佳}:\texttt{[0,1,2,3]}、\texttt{[2,3]}、\texttt{[0,2,3]} 等常 \textbf{≤ \textasciitilde 0.7 accuracy}(以相同的 train/test 集下測得)。
    \item \textbf{洞見}:再次驗證 \textbf{4–6} 的資訊性偏弱;若強行加入,反而稀釋辨識訊息。
\end{itemize}

\subsection{交叉驗證}

\textbf{完整管線重複 10 次},每次重新隨機劃分 Train/Test 後再做訓練與測試,最後彙整\textbf{平均結果}作為報告數字。

\section{設計抉擇與理由}

\subsection{為何 AE 而非 VAE}

目標是「重建誤差」本身;VAE 的機率建模與潛空間正則化在此非必要,AE 訓練更簡潔、較易穩定,且簡單測試之下VAE確實沒有優化分類結果。

\subsection{為何 1D-CNN(不做 FFT)}

再平層屬時域瞬時事件;1D-CNN 能直接學到局部時域形狀與跨通道關聯。頻域特徵未必優於學得的時域特徵,且 FFT 會引入額外窗口化選擇與相位處理成本,且簡單測試之下確實沒有優化分類結果。

\subsection{雙 AE + 2D loss 的好處}

相對「單一 AE + 閾值」,「Normal-AE vs Abnormal-AE」兩維 loss (如同二維平面) 讓邊界更清楚也更直觀(如:正常樣本在 Normal-AE 上 loss 低、在 Abnormal-AE 上 loss 高,反之亦然),線性或核化分類器即可取得乾淨決策邊界。

\section{侷限與風險}

\begin{itemize}
    \item \textbf{異常樣本偏少}(train 只有 62),泛化需小心;可以參考\textbf{分層 K-fold} 或\textbf{時間分層}的方式補強估計不確定性,或以\textbf{重抽樣/類別權重}平衡分類器。(此專案使用了SMOTE做法)
    \item \textbf{資料期間單一}:僅使用 0604–0607;需在長週期、跨多台電梯上做\textbf{時變漂移(concept drift)}評估與更新策略。
    \item \textbf{尚未落地}:流程未串上即時資料流與告警回饋迴路,須補上部屬與維運面。
\end{itemize}

\section{可落地化的下一步(Roadmap)}

\subsection{評估指標調校}

明確追蹤 \textbf{Recall / Precision / F1 / AUROC} 與 \textbf{告警頻率};以 ROC/PR-Curve 在不同閾值與分類器設定下找 sweet spot。

\subsection{模型管理}

每台電梯一組模型 vs 全域模型 + 微調;加上 \textbf{週/月回訓} 與 \textbf{資料漂移監測}(如 PSI、KS)。

\subsection{跨通道早期融合}

嘗試在 encoder 端做多通道卷積與注意力(仍保留後端 OR 作保險),觀察是否能在不吞噬 recall 的情況下提升 precision。

\section{研究貢獻說明}

\subsection{個人重點貢獻}

\textbf{說明:本人未參與實體電梯實驗設計與資料蒐集,以下貢獻皆基於使用學長姐蒐集的實驗資料進行的演算法開發與分析。}

\subsubsection{實驗數據完整性}
基於學長姐蒐集的實驗資料,進行完整的測試實驗,設計能提高準確率的演算法

\subsubsection{算法創新}
\begin{itemize}
    \item 早期單 autoencoder 階段的程式碼曾被學長使用,但後續開發完全獨立完成
    \item \textbf{雙 autoencoder} 架構設計與實作
    \item \textbf{2D 資料分類器}(SVM、Logistic Regression 等)的開發
    \item \textbf{OR operation} 融合策略的發想與實作
\end{itemize}

\subsubsection{實驗驗證}
期末前完成的良好異常檢測分類結果,並使用 cross validation 進行驗證

\subsubsection{報告撰寫}
基於參與專案的經驗與使用學長姐蒐集的實驗資料進行的分析,完成以上報告與實驗內容

\vspace{2cm}

\noindent 專題 GitHub 連結: \url{https://github.com/egger-meow/CS-Undergraduate-Project}

\vspace{3cm}

\section*{指導教授簽名確認}

\vspace{3cm}

\noindent 指導教授簽名:\underline{\hspace{5cm}} \hspace{2cm} 日期:\underline{\hspace{3cm}}

\end{document}
